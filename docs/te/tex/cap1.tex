%%%%%%%%%%%%%%%%%%%%%%%%%%%%%%%%%%%%%%%%%%%%%%%%%%%%%%%%%%%%%%%%%%%%%%%%%%%%%
% Chapter 1: Motivaci�n y Objetivos 
%%%%%%%%%%%%%%%%%%%%%%%%%%%%%%%%%%%%%%%%%%%%%%%%%%%%%%%%%%%%%%%%%%%%%%%%%%%%%%%

 Si simplemente se desea escribir texto normal en LaTeX,
 sin complicadas formulas matematicas o efectos especiales
 como cambios de fuente, entonces simplemente tiene que escribir
 en espaniol normalmente.\par
 Si desea cambiar de parrafo ha de dejar una linea en blanco o bien
 utilizar el comando \par
 No es necesario preocuparse de la sangria de los parrafos:
 todos los parrafos se sangraran automaticamente con la excepcion
 del primer parrafo de una seccion.
 Se ha de distinguir entre la comilla simple izquierda
 y la comilla simple derecha cuando se escribe en el ordenador.
 En el caso de que se quieran utilizar comillas dobles se han de
 escribir dos caracteres comilla simple seguidos, esto es,
 comillas dobles.
 Tambien se ha de tener cuidado con los guiones: se utiliza un unico
 guion para la separacion de silabas, mientras que se utilizan
 tres guiones seguidos para producir un guion de los que se usan
 como signo de puntuacion --- como en esta oracion.

%---------------------------------------------------------------------------------
\section{Secci�n Uno}
\label{1:sec:1}
  Primer p�rrafo de la primera secci�n.


%---------------------------------------------------------------------------------
\section{Secci�n Dos}
\label{1:sec:2}
  Primer p�rrafo de la segunda secci�n.

\begin{itemize}
  \item Item 1
  \item Item 2
  \item Item 3
\end{itemize}

