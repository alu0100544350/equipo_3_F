%%%%%%%%%%%%%%%%%%%%%%%%%%%%%%%%%%%%%%%%%%%%%%%%%%%%%%%%%%%%%%%%%%%%%%%%%%%%%%%
% Chapter 2: Fundamentos Te�ricos 
%%%%%%%%%%%%%%%%%%%%%%%%%%%%%%%%%%%%%%%%%%%%%%%%%%%%%%%%%%%%%%%%%%%%%%%%%%%%%%%

%++++++++++++++++++++++++++++++++++++++++++++++++++++++++++++++++++++++++++++++

En \LaTeX{}~\cite{Lamport:LDP94} es sencillo escribir expresiones
matemaicas como $a=\sum_{i=1}^{10} {x_i}^{3}$
y deben ser escritas entre dos simbolos \$.
Los super indices se obtienen con el simbolo \^{}, y
los sub indices con el simbolo \_.
Por ejemplo: $x^2 \times y^{\alpha + \beta}$.
Tambien se pueden escribir formulas centradas:
\[h^2=a^2 + b^2 \]
%++++++++++++++++++++++++++++++++++++++++++++++++++++++++++++++++++++++++++++++

\section{Primer apartado del segundo cap�tulo}
\label{2:sec:1}
  Primer p�rrafo de la primera secci�n.

\section{Segundo apartado del segundo cap�tulo}
\label{2:sec:2}
  Primer p�rrafo de la segunda secci�n.

